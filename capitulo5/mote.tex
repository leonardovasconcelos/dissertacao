\section{Mote}\label{sec:mote}

O termo \textit{mote} � utilizado para se referir ao dispositivo que � n� em uma rede de sensores sem fio, foi introduzido por pesquisadores da Universidade de Berkeley \cite{culler2002mica} no in�cio da d�cada de 90, assim, constantemente podemos encontrar refer�ncias como ``Berkeley Mote'' para designar estes dispositivos de rede, independente do fabricante. 

O MicaZ � um \textit{mote} dispon�vel comercialmente e amplamente utilizado em pesquisas acad�micas. Possui um processador de 4 MHz com 8 \textit{bits}, mem�ria de 128 kB. A diferen�a comparado a um computador com a capacidade similar do in�cio da d�cada de 1980, como o 8088, � o baixo consumo de energia que varia de 15 $\mu$A no modo de economia at� 8 mA em execu��o. Outro componente significativo � o r�dio, esses dispositivos possuem transmissores com frequ�ncia de 2.4 GHz e taxa de dados de 250 kbps, usam a estrat�gia de acesso ao meio CSMA/CA, conseguindo transfer�ncias de 40.000 bps.


\begin{table}[h]
\centering
\begin{tabular}{lcc}
\hline
\textbf{Item}         & \textbf{MicaZ}              & \textbf{Iris}\\
\hline
Processador           & ATMega 128L 8MHz            & ATmega1281 16MHz                                                                  \\
\textit{Flash}        & \multicolumn{2}{c}{128 kB}                                                                                                                                                           \\
Mem�ria Dados         & \multicolumn{2}{c}{512 kB}                                                                                                                                                           \\
EEPROM                & \multicolumn{2}{c}{4 kB}                                                                                                                                                             \\
ADC                   & \multicolumn{2}{c}{10 bit}                                                                                                                                                           \\
\textit{Chip} de R�dio         & CC2420                     & RF230                                                                             \\
Frequ�ncia do R�dio   & \multicolumn{2}{c}{2.4 GHz}                                                                                                                                                          \\
Taxa de Transfer�ncia & \multicolumn{2}{c}{250 kbps}                                                                                                                                                         \\
Voltagem de Opera��o  & \multicolumn{2}{c}{3,6 - 2,7}                                                                                                                                                        \\
Consumo de Energia    & \begin{tabular}[c]{@{}c@{}}RX 19,7 mA\\ TX 17,4 mA\\ \textless 15 $\mu$A (dormindo)\end{tabular} & \begin{tabular}[c]{@{}c@{}}RX 16 mA\\ TX 17 mA\\ 8 $\mu$A (dormindo)\end{tabular}\\
\hline
\end{tabular}
\caption{Especifica��es de \textit{hardware} dos \textit{motes}}
\label{tab:micazxiris}
\end{table}

A jun��o destas caracter�sticas, fornecem recursos que tornam o \textit{mote} uma boa op��o para diversas aplica��es, a capacidade de computa��o o consumo de energia trazem a flexibilidade para implanta��o destes equipamentos em RSSF, a Tabela \ref{tab:micazxiris} cont�m as especifica��es dos \textit{hardwares} utilizados. 
