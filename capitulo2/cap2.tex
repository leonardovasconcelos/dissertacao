\chapter{Trabalhos Relacionados} \label{cap:cap2}
\outlineon=0



Existem diversos protocolos de sincroniza��o de tempo dispon�veis para redes sensores sem fio, a maior parte deles seguem as abordagens descritas no cap�tulo anterior. Neste cap�tulo � apresentado uma breve explica��o de alguns protocolos proeminentes para sincroniza��o de redes sensores sem fio.  



\section{RBS}
% \outline{
%     � um m�todo no qual o receptor usa as transmiss�es da camada f�sica para comparar os rel�gios.
%     Usa o tempo de chegada do pacote como tempo de refer�ncia de sincroniza��o.
% }

Elson et al. \cite{Elson:SIGOPS02} propuseram o Reference Broadcast Synchronization (RBS) em 2002, sua principal inova��o era a utiliza��o de mensagens de refer�ncia por difus�o para a elimina��o do n�o determinismo de comunica��o relacionado ao tempo de envio e acesso ao meio, baseando no fato de que as mensagens de \textit{broadcast} chegam nos receptores praticamente ao mesmo tempo. 


O Algoritmo \ref{alg:rbs} exemplifica o funcionamento do RBS, usando um cen�rio com \textit{receivers} onde um n� � determinado como refer�ncia, ele ir� enviar por difus�o uma mensagem de requisi��o para os n�s ao seu alcance. Cada n� que recebe a requisi��o ir� guardar seu tempo local. Ap�s, os pares de receptores trocam seus \textit{timestamps}, eles podem comparar seus tempos e calcular a diferen�a entre seus rel�gios (\textit{offset}). O escorregamento do rel�gio ser� � calculado utilizando o m�todo dos m�nimos quadrados para encontrar uma estimativa da inclina��o ou escorregamento do \textit{clock} do outro n�.


\begin{algorithm}[H]
% \small
\setstretch{1}
\SetAlgoVlined
\DontPrintSemicolon
%\Inicio{
Primeiro, o transmissor envia uma mensagem de refer�ncia por \textit{broadcast}.\;
Cada \textit{receiver} registra seu tempo de quando recebeu a mensagem de refer�ncia.\;
Pares de \textit{receivers} trocam os seus registros de tempo de recebimento da mensagem.\;
O \textit{offset} entre um par de \textit{receivers} � a diferen�a entre seus rel�gios locais.\;
\caption{\texttt{RBS Passos\label{alg:rbs}}}
\end{algorithm}



O RBS tem um alto n�vel de economia de energia, devido a ele realizar o processo de sincroniza��o somente quando necess�rio. Este modelo de funcionamento � denominado sincroniza��o \textit{post-facto}, nele os n�s correm seu rel�gio naturalmente e os tempos dos desvios dos seus rel�gios s� ser�o calculados na ocorr�ncia de um evento importante, assim o \textit{timestamp} resultante � vinculado ao evento e n�o altera o rel�gio local. A corre��o do rel�gio � feita utilizando uma tabela com a informa��o dos valores de \textit{offset} e \textit{skew} de todos os pares ao seu alcance, e a cada mensagem recebida de outro \textit{receiver} o valor � convertido para uma escala de tempo local.

A sincroniza��o em cen�rios com multi-saltos � poss�vel de ser implementada utilizando o RBS. Neste ambiente, apenas um n� como refer�ncia pode n�o ser o suficiente para cobrir todo o dom�nio de abrang�ncia da rede, ent�o m�ltiplos n�s de refer�ncia podem ser utilizados, cada um com seu dom�nio de cobertura. Na verdade, a transmiss�o com v�rios saltos poderia acarretar um atraso na propaga��o de uma mensagem, neste caso para tratar este problema, a sincroniza��o de dois n�s em dois dom�nios diferentes � realizado por um terceiro n� localizado na intersec��o dos dom�nios.


Vantagens do RBS s�o:
\begin{itemize}
 \item Remove as duas maiores fontes de atraso, tempo de envio e tempo de acesso.
 \item Sincroniza��o \textit{post-facto} realiza ajustes somente quando necess�rio, diminuindo o custo de energia.
\end{itemize}


E as desvantagens s�o:
\begin{itemize}
 \item N� que envia o \textit{beacon} nunca � sincronizado. Impossibilita a utiliza��o em aplica��es que necessitem sincroniza��o do n� raiz.
 \item Em redes de um salto com $n$ n�s, este protocolo precisa trocar $O(n^2)$ mensagens. Em caso de redes com grandes vizinhan�as pode ter alto custo de computa��o.
 \item Alto n�mero de troca de mensagens do protocolo pode aumentar o seu tempo de converg�ncia.
 \item N�o � escal�vel, com o aumento do n�mero de n�s o seu desempenho cai de forma significativa.
\end{itemize}



 
\section{LTS}
\outline{
    Lightweight Tree-Based Synchronization.
}

O protocolo \textit{Lightweight Tree-Based Synchronization} (LTS) proposto por Van Greunen e Rabaey \cite{vanGreunen:WSNA03} apresenta duas abordagens, uma centralizada e outra distribu�da ambas multi-saltos. O algoritmo segue o modelo \textit{sender-receiver} ilustrado anteriormente na Figura \ref{fig:pairwise2}, com esquema de sincroniza��o \textit{pairwise} e precisa de apenas 3 mensagens para sincronizar um par de n�s.

A vers�o centralizada � uma extens�o do exemplo simples de um salto, ela � baseada em um n� de refer�ncia que � a raiz de uma �rvore que comporta todos os elementos da RSSF. O in�cio do algoritmo � realizado a constru��o de uma �rvore geradora $T$ que cont�m todos os n�s, a profundidade deve ser minimizada, pelo fato de cada n�vel da �rvore introduz um salto na rede e cada salto resulta em ac�mulo de erro das trocas de mensagens realizadas. Toda vez que o algoritmo � executado, a �rvore � reconstru�da. Uma vez constru�da a �rvore, o n� de refer�ncia inicia o processo realizando sincroniza��o \textit{pairwise} com cada um dos seus n�s filhos em $T$. Uma vez sincronizado os n�s filhos, eles repetem o processo com seus subsequentes at� que todos os n�s estejam sincronizados. O tempo para o algoritmo convergir � proporcional a profundidade da sua �rvore.

A vers�o distribu�da do LTS n�o faz uso da constru��o da �rvore geradora, a sincroniza��o n�o � mais responsabilidade apenas do n� de refer�ncia e sim dos pr�prios n�s. Neste modelo existe um ou mais n�s de refer�ncia, eles s�o requisitados pelos n�s sensores a qualquer momento que esses precisem sincronizar bem como a frequ�ncia de ressincroniza��o. Para os n�s determinarem suas taxas de sincroniza��o � preciso reunir os seguintes par�metros: precis�o necess�ria, n�mero de saltos at� o n� de refer�ncia, taxa de escorregamento do rel�gio e o tempo desde a �ltima sincroniza��o. S� ent�o ele calcula sua taxa. Desta forma, quando o n� determinar que precisa sincronizar far� solicita��o de ressincroniza��o ao n� de refer�ncia mais pr�ximo. 

As principais vantagens s�o:
\begin{itemize}
 \item Usa poucos recursos computacionais, como mem�ria e CPU.
 \item Suporta sincroniza��o \textit{post-facto}.
 \item Robusto a varia��o de canais, mudan�as na topologia, tamanho e mobilidade da rede.
 \item Na vers�o distribu�da, certos n�s necessitam sincronizar menos frequentemente.
\end{itemize}

As desvantagens s�o:
\begin{itemize}
 \item A precis�o da t�cnica diminui de forma linear em rela��o a profundidade da �rvore.
 \item Sens�vel a falhas do rel�gio ou informa��o errada da sub-rede.
 %\item Redes muitos esparsas resultam em �rvore profunda.
\end{itemize}




% Os autores mostram que a complexidade de comunica��o e a precis�o de uma sincroniza��o multi-hop � uma fun��o da constru��o e a 
% profundidade da �rvore de expans�o. Al�m disso, os autores mostram que a taxa de atualiza��o necess�ria de sincroniza��o multi-hop 
% pode ser vista como uma fun��o do rel�gio deriva ea exatid�o da sincroniza��o de single-hop.


\section{TPSN}
\outline{
    TPSN n�o elimina incerteza no sender, apenas minimiza o erro.
    Ele tenta reduzir este n�o determinismo utilizando timestamping da mensagem na camada MAC.
}

O TPSN (Timing-sync Protocol for Sensor Networks) introduzido por Ganeriwal et. al. em 2003 \cite{ganeriwal2003}, usa o modelo \textit{sender-receiver} e organiza a rede na estrutura de uma �rvore, um �nico n� sincroniza toda a rede, para reduzir incertezas relativas a acesso ao meio TPSN utiliza \textit{timestamp} na camada MAC. � dividido em duas fases descritas a seguir:

\textbf{Fase de Descoberta de N�vel:} Esta fase � respons�vel por criar uma topologia hier�rquica da rede, inicia no momento que a rede � ligada definindo um n� como $root$ e atribuindo seu n�vel como 0. O n� $root$ envia uma mensagem de descoberta chamada \texttt{level\_discovery} que cont�m seu n�vel e identificador, todos os n�s vizinhos  que recebem este pacote usam ele para identificar seu n�vel, somando 1 ao n�vel do pacote recebido. Ent�o estes vizinhos imediatos ao $root$ reenviam a mensagem de descoberta com seu pr�prio identificador e n�vel, este processo repete at� que eventualmente todos os n�s tenham identificado seu pr�prio n�vel. Caso um n� n�o tenha identificado seu n�vel, seja por problema de erro na troca de mensagem ou por ter ingressado na rede ap�s a fase de descoberta ter sido conclu�da, este n� pode enviar uma mensagem de \texttt{level\_request} e seus vizinhos ir�o responder com seus respectivos n�veis. Ent�o ele atribui seu n�vel como sendo 1 a mais do que o menor n�vel dentre os recebidos. Uma falha importante que dever ser mencionada � o caso da reelei��o do n� $root$, caso este venha a cair, nesta hip�tese um dos n�s do n�vel 1 � eleito e d� in�cio a uma nova fase de descoberta.



\textbf{Fase de Sincroniza��o:} Com a estrutura hier�rquica da rede criada na fase anterior, temos o in�cio da sincroniza��o propriamente dita, onde o TPSN usa sincroniza��o \textit{pairwise} entre as arestas e funciona de forma similar ao LTS, iniciando a partir do $root$ e avan�ando at� os n�veis mais externos da rede.







%There are many existing protocols [40-42,6] which follow the hierarchical structure to spread the reference clock value. Due to the hierarchical nature of the protocol, the synchronization error of a node with respect to the root node depends on its hop distance from the root node [43]. The synchronization time for the whole network increases as the number of levels in the hierarchy increases. As a result, it takes more time to reestablish the network hierarchy and synchronization over the whole network after any failure of the root node.












Vantagens do TPSN

\begin{itemize}
 \item in order to save energy, the protocol also supports \textit{post-facto}
 \item Any synchronization packet has the four delays discussed earlier: send time, access time, propagation time, and receive time. Eliminating any of these would be a plus. Although TPSN does not eliminate the uncertainty of the sender it does, however, minimize it. Also, TPSN is designed to be a multi-hop protocol; so transmission range is not an issue.

 \item Unlike RBS, TPSN has uncertainty in the sender. They attempt to reduce this non-determinism by time stamping packets in the MAC layer. It is claimed that the sender's uncertainty contributes very little to the total synchronization error. By reducing the uncertainty with low level time stamping, it is claimed that TPSN has a 2 to 1 better precision than RBS and that the sender to receiver synchronization is superior to the receiver to receiver synchronization. [Ganeriwal2003]

 \item RBS also is limited by the transmission range. It was stated that RBS can ignore the propagation time if the range of transmission was relatively small. If it is a large multi-hop network, this is not the case. RBS would have to send more reference beacons for the node to synchronize. TPSN on the other hand was designed for multi-hop networks. Their protocol uses the tree based scheme so the timing information can accurately propagate through the network.

 \item The sender to receiver synchronization method is claimed to be more precise than the receiver to receiver synchronization. Also TPSN is designed for multi-hop networks, where RBS works best on single hop networks. So, the transmission range is not a factor with TPSN.
\end{itemize}

Desvantagem do TPSN

\begin{itemize}
 \item ...
\end{itemize}





 
 
\section{PulseSync}

 \outline{

    No trabalho \cite{Lenzen:Sensys09} � realizado um estudo sobre o FTSP e � observado que o protocolo apresenta um crescimento do erro de sincroniza��o baseado no di�metro da rede.
    Ent�o prop�e o protocolo PulseSync para melhorar esse cen�rio.
    Tamb�m � dependente de MAC timestamping.
}

Lenzen \textit{et al.} em \cite{Lenzen:Sensys09} introduziram o PulseSync, e expandiram em \cite{lenzen2015pulsesync}. A ideia b�sica do PulseSync � distribuir a informa��o dos valores dos rel�gios da forma mais r�pida poss�vel usando o menor n�mero de mensagem. Na inicializa��o da rede os n�s escutam o canal esperando por mensagens de sincroniza��o, no em tando, passado um certo per�odo de tempo se n�o houver nenhum recebimento ou se ele tem o ID menor o n� se declara como \textit{root} e come�a a enviar periodicamente mensagens que s�o chamadas de pulsos. Esses pulsos s�o mensagens de \textit{broadcast} com as seguintes informa��es: \textit{timestamp}, n�mero de sequencia e o ID do \textit{root}.

Depois de passar da fase inicial, onde � definido o \textit{root}, os n�s vizinhos a ele come�ar�o a receber mensagens de sincroniza��o. Os rel�gios dos n�s s�o sincronizados atrav�s da rede utilizando \textit{root} como n� de refer�ncia, uma vez que um n� adjacente tenha recebido uma mensagem de sincroniza��o ele encaminha essa mensagem adiante. 

Em uma topologia em que determinado n� tenha muitos vizinhos, ele deve receber mensagens de sincroniza��o repetidas, ent�o ele s� ir� utilizar e encaminhar a que chegou primeira, pois essa provavelmente tem o menor caminho at� o \textit{root}, logo sofreu menos com o atraso introduzido pelo percurso da mensagem. Como forma de manter o menor n�mero de saltos o Pulsesync utiliza busca em largura (Breadth-First Search - BFS).


Como forma de diminuir os atrasos referentes a comunica��o, o Pulsesync utiliza \textit{timestamp} na camada MAC. Para transmitir rapidamente os pulsos atrav�s da rede, os n�s come�am a retransmitir as mensagens t�o logo elas cheguem, o que pode causar colis�es devido as interfer�ncias do meio sem fio, existem algumas t�cnicas que trabalham em como melhorar a difus�o de mensagens por inunda��o evitando a interfer�ncia \cite{levis2003trickle, stann2006rbp, zhu2010exploring}, o Pulsesync n�o implementa essas solu��es apenas define o intervalo de separa��o dos pulsos para evitar a colis�es. Para corre��o do escorregamento do rel�gio � utilizado regress�o linear para corrigir o declive do crescimento do rel�gio. O pseudoc�digo c�digo do n� \textit{root} e do n� cliente � descrito pelos Algoritmos \ref{alg:pulseroot} e \ref{alg:pulsenonroot}, respectivamente.


\begin{algorithm}[H]
% \small
\KwData{R(t) = rel�gio local}
\setstretch{1}
\SetAlgoVlined
\DontPrintSemicolon
%\Inicio{
% \While{R(t) mod(B) \equal 0 }{
 \Se{Root}{
   Aguarda intervalo de sincroniza��o\;
   Envia(R(t), $seqNum$)\;
   $seqNum \leftarrow seqNum$ + 1 \;
 }
% }
\caption{\texttt{Rotina peri�dica sempre no intervalo do \textit{Beacon} B\label{alg:pulseroot}}}
\end{algorithm}



\begin{algorithm}[H]
% \small
\setstretch{1}
\SetAlgoVlined
\DontPrintSemicolon
%\Inicio{
 \Se{N�o Root}{
    Recebe pulso\;
    Armazena(pulso)\;
    Delete(entradas velhas)\;
    Aguarda \textit{backoff} para reencaminhar pulso\;
    Envia(R(t))\;
    $seqNum \leftarrow seqNum + 1 $
  }
\caption{\texttt{N� n�o \textit{root}\label{alg:pulsenonroot}}}
\end{algorithm}




Suas vantagens s�o:

\begin{itemize}
 \item Algoritmos assintoticamente �timo, tem converg�ncia r�pida.
 \item Apresenta bom custo benef�cio entre escalabilidade e efici�ncia energ�tica.
 \item Diminuiu a propaga��o de erro em redes multi-saltos.
\end{itemize}



Suas desvantagens s�o:

\begin{itemize}
 \item Dependente de \textit{hardware} espec�fico para \textit{timestamp} na camada MAC.
 \item N� \textit{root} � um ponto �nico de falha.
\end{itemize}



 
\section{FTSP}

\outline{
    Descri��o de alto n�vel sobre o FTSP.
}

mestrado de pt:


O Flooding Time Synchronization Protocol (FTSP) \cite{maroti2004} � um protocolo desenvolvido para ser uma solu��o de sincroniza��o de RSSF que atenda as necessidades de escalabilidade, robustez a mudan�as na topologia da rede relacionadas a falhas de n�s, enlace e mobilidade.  

The goals of the Flooding Time Synchronization Protocol (FTSP) (Mar�ti et al. 2004) are to achieve network-wide synchronization with errors in the microsecond range, scalability up to hundreds of nodes, and robustness to changes in network topology including link and node failures. FTSP differs from other solutions in that it uses a single broadcast to establish synchronization points between sender and receivers while eliminating most sources of synchronization error. 

Toward this end, FTSP expands on the delay analysis described in Section XX and decomposes the end-to-end delay into the components shown in Figure X. In this analysis, the wireless radio of the sensor node informs the CPU using an interrupt at time $t_1$ that it is ready to receive the next piece of the message to be transmitted. After the interrupt handling time d 1 , the CPU generates a time stamp at time $t_2$ . The time needed by the radio to encode and transform the piece of the message into electromagnetic waves is described as encoding time $d_2$ (between $t_1$ and $t_3$ ). The propagation delay (between $t_3$ on node j's clock and $t_4$ on node k's clock) is followed by the decoding time $d_4$ (between $t_4$ and $t_5$ ). This is the time the radio requires to decode the message from electromagnetic waves back into binary data. The byte alignment time $d_5$ is a delay caused by the different byte alignments (bit offsets) of nodes j and k, that is, the receiving radio has to determine the offset from a known synchronization byte and then shift the incoming message accordingly. Finally, the radio on node k issues an interrupt
at time $t_6$ , which allows the CPU to obtain a final time stamp at time $t_7$ .


\begin{figure}[h]
 \centering
 \includegraphics{./figuras/exemplo_ftps1.eps}
 % exemplo_ftps1.eps: 0x0 pixel, 300dpi, 0.00x0.00 cm, bb=0 0 330 180
 \caption{Exemplo do fluxo da rede com FTSP}
 \label{fig:exemplo_ftsp1}
\end{figure}



philipe aqui:
This shortcoming is tackled by the Flooding-Time Synchronization Protocol (FTSP) [19]. FTSP elects a root node based on the smallest node identifier and forms an ad-hoc tree structure by flooding the current time information of the root node into the network. Each node uses a linear regression table to convert between the local hardware clock and the clock of the reference node. 



As suas vantagens s�o:
\begin{itemize}
 \item O algoritmo estado da arte em sincroniza��o de rel�gios em redes sensores sem fio.
 \item Resiliente as mudan�as da topologia da rede.
 \item Converg�ncia r�pida do tempo global.
 \item Regress�o linear para tratar o escorregamento do rel�gio.
 \item Algoritmo simples de elei��o de l�der.
\end{itemize}

As suas desvantagens s�o:
\begin{itemize}
 \item Exibe crescimento de erros baseado no tamanho da rede em saltos.
 \item Dependente de \textit{hardware} espec�fico para funcionamento do seu \textit{timestamp}.
\end{itemize}




\begin{figure}[h]
 \centering
 \includegraphics{./figuras/classificacao_protocolos.eps}
 % classificacao_protocolos.eps: 0x0 pixel, 300dpi, 0.00x0.00 cm, bb=
 \caption{Classifica��o dos protocolos de sincroniza��o}
 \label{fig:classificacao_prot}
\end{figure}

A Figura \ref{fig:classificacao_prot} separa os protocolos descritos neste cap�tulo baseado nos modelos de troca de mensagens descritos na Se��o \ref{subsec:msg_sinc}. No pr�ximo cap�tulo teremos a descri��o mais detalhada do funcionamento do FTSP, que � o protocolo e principal fonte de interesse deste trabalho.