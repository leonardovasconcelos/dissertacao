\chapter{Trabalhos Relacionados} \label{cap:cap2}
\outlineon=1
asdf
\outline{
\begin{itemize}
 \item Intro sobre redes sensores sem fio.
 \item Necessidade da Sincroniza��o em sistemas distribu�dos.
 \item Problemas inerentes a sincroniza��o, fontes de atraso e imprecis�o em RSSF
\end{itemize}
}

\section{Rel�gios e o Problema de Sincroniza��o}


\section{Sincroniza��o em Redes Sensores Sem Fio}


\section{B�sico de Sincroniza��o de Tempo}

\subsection{Mensagens de Sincroniza��o}

\subsection{Fontes de Imprecis�o em Comunica��o}

\section{Protocolos de Sincroniza��o}\label{sec:into}

Falar sobre 


\subsection{RBS}
\outline{
    � um m�todo no qual o receptor usa as transmiss�es da camada f�sica para comparar os rel�gios.
    Usa o tempo de chegada do pacote como tempo de refer�ncia de sincroniza��o.
}

\subsection{LTS}
\outline{
    Lightweight Tree-Based Synchronization.
}

\subsection{TPSN}
\outline{
    TPSN n�o elimina incerteza no sender, apenas minimiza o erro.
    Ele tenta reduzir este n�o determinismo utilizando timestamping da mensagem na camada MAC.
}
 
\subsection{PulseSync}

 \outline{

    No trabalho \cite{Lenzen:Sensys09} � realizado um estudo sobre o FTSP e � observado que o protocolo apresenta um crescimento do erro de sincroniza��o baseado no di�metro da rede.
    Ent�o prop�e o protocolo PulseSync para melhorar esse cen�rio.
    Tamb�m � dependente de MAC timestamping.
}


