\section{Objetivos da Disserta��o}\label{sec:contrib}
\outline{
\begin{itemize}
 \item Fazer o FTSP funcionar com timestamp a n�vel de aplica��o.
\end{itemize}
}
 
Baseado no problema descrito na Se��o \ref{sec:problema}, este trabalho prop�e o FTSP+ a extens�o do FTSP para funcionamento independente de \textit{timestamp} na camada de acesso ao meio. O objetivo � fornecer uma extens�o para o funcionamento do FTSP em plataformas que n�o suportam a implementa��o do \textit{timestamp} na camada MAC, visto que ainda � poss�vel fornecer a funcionalidade de sincroniza��o a n�vel de pacote, com um baixo custo de \textit{overhead} de comunica��o. Al�m de manter as qualidades do protocolo, como, alta precis�o, compensa��o do rel�gio e algoritmo de elei��o de l�der robusto. Realizar a implementa��o do FTPS+ de acordo com o padr�o de sincroniza��o formalizado pelo sistema TinyOS em sua documenta��o \cite{tinyos133}. 

%TEPs s�o propostas de aprimoramento, documentos que normalizam as modifica��es e estrutura do c�digo do TinyOS.
