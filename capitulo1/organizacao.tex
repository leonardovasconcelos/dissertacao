\section{Organiza��o do Documento}\label{sec:org}

Esta disserta��o � composta por sete cap�tulos, os assuntos discutidos neste trabalho est�o organizados conforme mostrado a seguir.



\begin{itemize}
 \item \textbf{Cap�tulo 2:} Apresenta e descreve os principais protocolos de sincroniza��o, detalhando seu funcionamento principalmente ao que tange os seus mecanismos de mensagens e corre��o de rel�gio, suas vantagens e desvantagens. No final traz uma classifica��o dos protocolos os agrupando de acordo com seus modelos de troca de mensagens.
 
 \item \textbf{Cap�tulo 3:} � realizado o detalhamento do protocolo FTSP, que serviu de base para a implementa��o da proposta apresentada nesta disserta��o. Detalha-se ainda seus principais componentes com o foco de posteriormente mostrar as modifica��es.
 
 \item \textbf{Cap�tulo 4:} Apresenta o FTSP+, descreve o conceito da t�cnica que faz a estimativa de atraso que possibilita a substitui��o do MAC \textit{timestamp}, por um a n�vel de aplica��o. 
 
 \item \textbf{Cap�tulo 5:} Apresentamos as ferramentas de software e os conceitos por traz do desenvolvimento do FTSP+. O sistema operacional TinyOS e a linguagem de programa��o nesC.
 
 \item \textbf{Cap�tulo 6:} Neste cap�tulo, descrevemos os experimentos realizados e os valores, m�tricas, ambientes e resultados.
 
 \item \textbf{Cap�tulo 7:} Conclus�o acerca do trabalho baseado nos experimentos e resultados, trata do produto da disserta��o e a aplica��o da t�cnica em trabalhos futuros.
\end{itemize}
 
