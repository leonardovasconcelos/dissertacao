\section{Motiva��o do Trabalho}\label{sec:motivacao}

\outline{
\begin{itemize}
   \item Devido a diversidade de dispositivos e o bom desempenho do FTSP, tona-lo um protocolo independente de recurso de hardware possibilitando o seu funcionamento em \textit{motes} sem o recurso de MAC timestamp.
\end{itemize}
}

Redes sensores s�o compostas de um grande n�mero de n�s. Assim, torna-se fundamental que o pre�o de um �nico sensor n�o seja muito alto, para n�o comprometer o valor da rede inteira, o que pode inviabilizar a implanta��o de um projeto. Neste sentido, os componentes do n� sensor podem se tornar cada vez mais simples ou menos especializados, por exemplo, o transmissor de r�dio. Este componente pode n�o contar com uma interface t�o rica de recursos como MAC \textit{timestamp} e acesso rand�mico ao FIFO durante a transmiss�o.

Diante das restri��es originais apresentadas pela implementa��o padr�o do protocolo FTSP, no que tange a exig�ncias espec�ficas de \textit{hardware}, al�m da imposi��o de modelos restritivos de sincroniza��o inerente as RSSF, buscou-se uma abordagem que pudesse ser livre de requisitos espec�ficos e simples de acoplar ao sistema.
Uma abordagem de sincroniza��o independente do recurso de MAC \textit{timestamp} para o FTSP poderia torn�-lo apropriado para uma maior quantidade de dispositivos.