\section{Defini��o do Problema}\label{sec:problema}

\outline{Depend�ncia de hardware, padroniza��o do recurso, esfor�o para que exista futuramente um padr�o que esteja presente em todos os tipos de \textit{motes} e possibilite assim a interopera��o de protocolos de MAC \textit{timestamp} em sistemas heterog�neos.}

Alguns tipos de r�dios tem embutido \textit{chips} mais sofisticados capazes de coordenar a transmiss�o sem sobrecarga da CPU.
Esses r�dios tem a capacidade de acessar a fila de transmiss�o a qualquer momento, por exemplo, um dos \textit{chips} mais comuns aplicados a RSSF o \textit{Chipcon} CC2420 \cite{chipcon20032} que utiliza o padr�o IEEE 802.15.4  tem essa caracter�stica, ele tem um recurso chamado SFD (Start of Frame Delimiter) que acessa a fila de transmiss�o e permite gerar interrup��es durante o envio e recebimento dos dados, possibilitando o seu emprego em mensagens de sincroniza��o mais refinadas realizadas na camada MAC. 

A marca de tempo das mensagens na camada de acesso ao meio reduz o atraso relacionados as incertezas de comunica��o, neste modelo � poss�vel inserir o \textit{timestamp} na mensagem no momento da transmiss�o e imediatamente no recebimento, com o \textit{MAC timestamp} � poss�vel eliminar as fontes de atraso na entrega da mensagem.

O FTSP � o estado da arte dos protocolos de sincroniza��o \textit{multi-hop} \cite{Sommer:IPSN09}, e um dos mais populares \cite{lenzen2015pulsesync,holtkamp2013,shannon2012dynamic} e vem dispon�vel por padr�o no sistema operacional TinyOS. Uma de suas principais caracter�sticas � o uso de timestamp na camada MAC. Se uma plataforma particular n�o suportar a mudan�a do conte�do da mensagem depois de ocorrido o evento de sincroniza��o, ent�o o FTSP n�o pode ser implementado nessa plataforma. Assim o protocolo � fortemente dependente de recursos espec�ficos no \textit{chip} de r�dio. 
