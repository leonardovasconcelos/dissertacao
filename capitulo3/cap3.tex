\chapter{Revis�o detalhada do FTSP} \label{cap:cap3}
\outlineon=1

\outline{
\begin{itemize}
  \item Explicar a t�cnica.
  \item Inserir ilustra��o do processo de sincroniza��o com o rel�gio externo. 
  \item Pseudo c�digo
  \item FTSP+ e subsection
\end{itemize}
}


\begin{table}[htb!]
 \centering
 \begin{tabular}{c|c|c}	
 	\multirow{2}{*}{\textbf{Tempo}}	& \multirow{2}{*}{\textbf{Magnitude}}	& \multirow{2}{*}{\textbf{Distibui��o}} \\
		&&\\
		\hline
      \multirow{2}{*}{ $\mathbf{Enviar~e~Receber}$ } 			&  \multirow{2}{*}{$0 - 100~ms$}	& N�o determin�stico,	\\
		&& depende da carga do CPU \\
		\hline
      \multirow{2}{*}{$\mathbf{Acesso~ao~Meio}$}  			&  \multirow{2}{*}{$10~ms - 500~ms$}	& N�o determin�stico,	\\
		&& depende da conten��o do canal \\
		\hline
      $\mathbf{Transmiss�o~e}$ 			& \multirow{2}{*}{$10~ms - 20~ms$}	& Determin�stico, depende	\\
		$\mathbf{Recep��o}$&& do tamanho da mensagem \\
		\hline
      \multirow{2}{*}{$\mathbf{Propaga��o}$} 				& $< 1\mu s$ para 			& Determin�stico,	\\
		& dist�ncias acima de 300m & depende da dist�ncia\\
		\hline
      \multirow{2}{*}{$\mathbf{Tratar~Interrup��o}$} 			& $< 5\mu s$ na maioria dos casos, 	& N�o determin�stico,	\\
		&mas pode chegar a $30~\mu s$& depende das interrup��es\\ 
		\hline
      $\mathbf{Codificar}$ 		& $100~\mu s - 200\mu s$,		& Determin�stico,	\\
		$\mathbf{Decodificar}$&$< 2\mu s$ de vari�ncia& depende o chip de r�dio \\
		\hline
      $\mathbf{Alinhamento~de}$ 			& \multirow{2}{*}{$0 - 400\mu s$} 	& Determin�stico,	\\
		$\mathbf{Bytes}$&& pode ser calculado
      
 \end{tabular}
 \caption{Fontes de atraso na transmiss�o de mensagens \cite{maroti2004}}
 \label{tab:mean_sd}
\end{table}
 
 

\section{\textit{Timestamp}}

\outline{
  Explicar como funciona a marca��o de tempo no FTSP
}


\section{Escorregamento do Rel�gio}

\outline{
  Descrever os problemas relacionados ao clock drift.
  Como o FTSP corrige essa fonte de imprecis�o.
}


\section{Sincroniza��o Multi-saltos}

\outline{
\begin{itemize}
 \item Formato da mensagem de sincroniza��o.
 \item Gerenciamento de informa��o redundante.
 \item Elei��o do n� raiz.
 \item Evento: enviar/receber mensagem de sincronia. 
\end{itemize}

}