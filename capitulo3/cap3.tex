\chapter{Revis�o detalhada do FTSP} \label{cap:cap3}
\outlineon=0

\outline{
\begin{itemize}
  \item Explicar a t�cnica.
  \item Inserir ilustra��o do processo de sincroniza��o com o rel�gio externo. 
  \item Pseudo c�digo
  \item FTSP+ e subsection
\end{itemize}
}


Neste cap�tulo iremos descreve-se o funcionamento do FTSP, principalmente no que se refere aos esquemas de sincroniza��o com o n� \textit{root}, fase de elei��o e reelei��o do l�der, envio peri�dico de \textit{beacons}, marca��o de tempo na camada de acesso ao meio e o uso de regress�o linear para corrigir o escorregamento do rel�gio. 


\section{Flooding Time Synchronization Protocol}\label{sec:ftsp}


O FTSP surgiu em 2004 \cite{maroti2004} tendo como principal meta efetuar a sincroniza��o dos rel�gios de todos os participantes de uma rede, oferecendo boa performance, simplicidade e baixo custo computacional. Essas caracter�sticas visam atender as restri��es que os componentes das RSSF possuem, onde todos os n�s sofrem de erros ocasionados pela impureza do cristal, bem como, erros inerentes ao enlace de comunica��o sem fio n�o confi�vel.

A partir do uso de uma �nica mensagem o FTSP sincroniza m�ltiplos receptores. Neste processo, a mensagem com a marca de tempo � gravada no instante em que est� sendo enviada, que tamb�m � praticamente o mesmo momento em que esta sendo marcada como recebida pelo receptor. Este mecanismo de \textit{timestamp} na camada MAC elimina a maior parte das fontes de erros na etapa de comunica��o comentados na Se��o \ref{subsec:fontes_imp} e � empregado em muitos outros protocolos de sincroniza��o \cite{Ganeriwal:Sensys03, Lenzen:Sensys09}. O trabalho \cite{elson2002} trata do uso de regress�o linear para a corre��o do escorregamento do rel�gio, que � o usado pelo FTSP para compensar o seu rel�gio e assim manter um alto n�vel de acur�cia.

O n� \textit{root} � o respons�vel por disseminar o tempo global pela rede, por�m h� redes em que os n�s n�o est�o ao alcance da faixa de cobertura do seu sinal. O FTSP resolve esse problema, pois disp�e do recurso de rede multi-saltos, pelo qual constr�i uma estrutura \textit{ad hoc}. Nessa nova estrutura, os n�s diretamente sincronizados com o \textit{root} assim que ficam atualizados passam a sincronizar os n�s subsequentes com sua informa��o de tempo global. 


\section{\textit{Timestamp}}

\outline{
  Explicar como funciona a marca��o de tempo no FTSP
}

%O \textit{timestamp} na camada de acesso ao meio � utilizado para reduzir o atraso gerado na etapa de comunica��o do processo de sincroniza��o.

O FTSP usa \textit{broadcast} para sincronizar seus n�s, essa mensagem cont�m um \textit{timestamp} com o valor estimado do tempo global no momento em que a transmiss�o � realizada. No instante do recebimento os n� obt�m o valor do tempo do seu rel�gio local, assim com a diferen�a entre o tempo global e o local o n� pode estimar o seu \textit{offset}. 

Caso a leitura do tempo seja armazenada na etapa de envio ainda no n�vel de aplica��o, v�rios componentes de atrasos ser�o acumulados at� que a mensagem seja lida pelos receptores, a Se��o \ref{subsec:fontes_imp} lista esses componentes de atraso e a Tabela \ref{tab:decomp} quantifica a magnitude dessas fontes. A Figura \ref{fig:comp_apptimestamp} ilustra esse procedimento como segue, no tempo $t_0$ � iniciado o processo de envio da mensagem de sincroniza��o a partir do n� $i$, logo em seguida � criado o pacote que armazena o \textit{timestamp} $t_1$ e comanda o envio para o sistema operacional, este encaminha a mensagem para a pilha de protocolos. O tempo de acesso ao meio � vari�vel e depende da janela de conten��o da rede. Quando $i$ ganha o direito de enviar a mensagem, d�-se inicio a transmiss�o em $t_2$, $t_3$ o n� $j$ termina de receber a transmiss�o, mas somente algum tempo depois o no $j$ gera uma interrup��o e o SO ir� armazenar o \textit{timestamp} da recep��o em $t_5$. Neste momento $j$ tem como calcular seu \textit{offset}, por�m esse valor n�o � representativo devido aos atrasos que as marcas de tempo sofreram em rela��o de seus valores ideais.

%Para fazer isso o FTSP conta com o uso do \textit{timestamp} na camada MAC, que funciona da seguinte forma 

\begin{figure}[h]
 \centering
 \includegraphics{./figuras/comparacao_mactime.eps}
 % comparacao_mactime.eps: 0x0 pixel, 300dpi, 0.00x0.00 cm, bb=
 \caption{Procedimento de leitura do \textit{timestamp} em n�vel de aplica��o}
 \label{fig:comp_apptimestamp}
\end{figure}

Como forma de corrigir esse problema a leitura dos tempos podem ser efetuadas 



Quando um comando de transmiss�o � emitido, a mensagem de sincroniza��o est� devidamente estabelecida e em fila para transmiss�o. No entanto, quando - na camada f�sica - a transmiss�o do primeiro bit � iniciado, "�ltimo minuto" tempo de estampagem � executada. Em um local previamente reservados dentro da mensagem um valor de tempo � inserido enquanto o in�cio da mensagem j� est� a ser transmitido. Isto � feito na mem�ria do chip de r�dio. O registro de tempo permite gravar informa��es adicionais na mensagem, sobre quanto tempo a cria��o e o processamento da mensagem levou o transmissor. Por sua vez, isso permite que todos os receptores desta mensagem para calcular quando o comando de envio foi emitido e elimina os atrasos de remetente, que de outra forma representam o limite de precis�o.


% Several transceivers allow for modifying the contents of a packet after packet transmission is started. Packet-level time synchronization can be implemented very efficiently on such platforms.
% 
% Transmitter's story
% 
% When the communications stack services a TimeSyncAMSend.send command called with event timestamp t\_e, it stores t\_e (e.g. in a map with the pointer of the message\_t as key) and sets the designated timestamp field in the packet payload to 0x80000000.
% When the packet starts being transmitted over the communication medium, a corresponding hardware event is timestamped (e.g. an SFD interrupt). Let us denote this transmission timestamp with t\_tx. The difference of event timestamp t\_e and transmit timestamp t\_tx is written into the designated timestamp field in the payload of the packet (typically into the footer, since the first few bytes might have been transmitted by this time). That is, the information the packet contains at the instance when being sent over the communications medium is the age of the event (i.e. how much time ago the event had occurred).
% If an error occurs with timestamping the transmission or with writing the package payload after transmission has started, then the designated timestamp field in the packet payload will contain 0x80000000, indicating the error to the receiver.
% Receiver's story
% 
% The packet is timestamped with the receiver node's local clock at reception (e.g. with the timestamp of the SFD interrupt). Let us denote the time of reception with t\_rx. The reception timestamp is stored in the metadata structure of the message\_t [5].
% When the event time is queried via the TimeSyncPacket interface, the eventTime command returns the sum of the value stored in the designated timestamp field in packet payload and the reception timestamp, i.e. e\_t- e\_tx+e\_rx. This value corresponds to the time of the event in the receiver's local clock.
% The TimeSyncPacket.isValid command returns FALSE if the time value stored in the payload equals 0x80000000 or if the communications stack failed to timestamp the reception of the packet. Otherwise TRUE is returned, which indicates that the value returned by TimeSyncPacket.eventTime can be trusted.






\begin{figure}[h]
 \centering
 \includegraphics{./figuras/comparacao_mactime2.eps}
 % comparacao_mactime2.eps: 0x0 pixel, 300dpi, 0.00x0.00 cm, bb=
 \caption{Procedimento de leitura do \textit{timestamp} na camada MAC}
 \label{fig:comp_mactimestamp}
\end{figure} 


\section{Escorregamento do Rel�gio}

\outline{
  Descrever os problemas relacionados ao clock drift.
  Como o FTSP corrige essa fonte de imprecis�o.
}


\begin{figure}[h]
 \centering
 \includegraphics{./figuras/grafico_regressao.eps}
 % grafico_regressao.eps: 0x0 pixel, 300dpi, 0.00x0.00 cm, bb=
 \caption{Regress�o linear aplicada a \textit{timestamps}}
 \label{fig:regressao}
\end{figure} 


\section{Sincroniza��o Multi-saltos}

\outline{
\begin{itemize}
 \item Formato da mensagem de sincroniza��o.
 \item Gerenciamento de informa��o redundante.
 \item Elei��o do n� raiz.
 \item Evento: enviar/receber mensagem de sincronia. 
\end{itemize}

}




\begin{figure}[h]
 \begin{Verbatim}[numbersep=1pt,frame=single]
typedef nx_struct TimeSyncMsg
{
 nx_uint16_t rootID;
 nx_uint16_t nodeID; 
 nx_uint8_t  seqNum;
 nx_uint32_t globalTime;
 nx_uint32_t localTime;
} TimeSyncMsg;
\end{Verbatim}
\caption{Formato da mensagem de sincroniza��o}
\end{figure}




%This section briefly describes the Flooding Time Synchronization Protocol (FTSP). For more detailed information, refer to \cite{maroti2004}.

% FTSP is a synchronization protocol for WSN that provides high accuracy, consumes few resources, uses little bandwidth and is fault-tolerant. It elects a node as root to provide the time reference for synchronization; if root failure is detected (using timeouts), another root is elected.
% Root and synchronized nodes send synchronization messages periodically, and receiving nodes use these messages to synchronize.
% Therefore, FTSP supports multi-hop networks.
% 
% Synchronization messages comprise a \textit{sender timestamp} which is the estimated global time and \textit{rootID} which is the network identifier of the root (where the node with the lowest ID is the chosen root).
% \textit{seqNum} is a sequence counter that is incremented each synchronization round; this field is used to verify the redundancy of messages \cite{maroti2004}.
% 
% All nodes think they are root when the network starts, so they broadcast synchronization messages to the network. 
% When they receive a synchronization message, they check who has the lowest ID: if the local ID is higher, this node gives up on being root and starts synchronizing.
% Another important check is the \textit{seqNum}.
% If it is greater than the local value \textit{highestSeqNum}, it means that this is a new synchronization message and starts the synchronization procedure.
% 
% The synchronization procedure consists of computing a linear regression \cite{elson2003} that will provide the clock skew (used to estimate the global time) in relation to the reference node.
% The last step is to forward its local (synchronized) time to other nodes. 

\begin{figure}[h]
 \centering
 \includegraphics{./figuras/exemplo_ftps1.eps}
 % exemplo_ftps1.eps: 0x0 pixel, 300dpi, 0.00x0.00 cm, bb=0 0 330 180
 \caption{Exemplo do fluxo da rede com FTSP}
 \label{fig:exemplo_ftsp1}
\end{figure}


\begin{figure}[h]
\begin{Verbatim}[numbers=left,numbersep=1pt,frame=single]
event Radio.receive(TimeSyncMsg *msg){
  if( msg->rootID < myRootID )
      myRootID = msg->rootID;
  else if( msg->rootID > myRootID
    || msg->seqNum <= highestSeqNum )
      return;
  highestSeqNum = msg->seqNum;
  if( myRootID < myID )
      heartBeats = 0;
  if( numEntries >= NUMENTRIES_LIMIT
    && getError(msg) > TIME_ERROR_LIMIT )
      clearRegressionTable();
  else
      addEntryAndEstimateDrift(msg);
}
\end{Verbatim} 
 \caption{FTSP Receive Routine \cite{maroti2004} }%\vspace{0em}
 \label{figure1}
\end{figure}

% Figure 1 shows the routine of receiving synchronization messages.
% Lines 2 and 3 compare the \textit{rootID} of the synchronization message with the local \textit{rootID}.
% If the message has a lower \textit{rootID}, the node assumes this \textit{rootID} as root.
% Lines 4 to 7 ignore messages with higher  \textit{rootID} and lower \textit{seqNum}.
% If \textit{seqNum} is higher, local \textit{highestSeqNum} is updated.
% Lines 8 and 9 makes a root node give up on being root when it has a \textit{rootID} lower than its own ID.
% In case the \textit{rootID} is larger it is checked whether the \textit{seqNum} is greater or equal to the value of \textit{highestSeqNum}, this check prevents information redundancy because the message will only be used when the \textit{rootID} is less than or equal to \textit{myRootID} and the number greater than the value of \textit{highestSeqNum}.
% Lines 10 to 15 verify if the time of a message is in disagreement with the earlier estimates of global time, if applicable clear the regression table, if not, accumulate synchronization messages to calculate the linear regression and synchronize.


%Lines 10 to 15 accumulate synchronization messages to calculate the linear regression and synchronize.



\begin{figure}[h]
   \begin{Verbatim}[numbers=left,numbersep=1pt,frame=single]
event Timer.fired() {
  ++heartBeats;
  if( myRootID != myID
    && heartBeats >= ROOT_TIMEOUT )
      myRootID = myID;
  if( numEntries >= NUMENTRIES_LIMIT
    || myRootID == myID ){
      msg.rootID = myRootID;
      msg.seqNum = highestSeqNum;
      Radio.send(msg);
  if( myRootID == myID )
      ++highestSeqNum;
  }
}
  \end{Verbatim}
  \caption{FTSP Send Routine \cite{maroti2004}}
  \label{figure2}
\end{figure}

% Figure \ref{figure2} shows the sending routine.
% A node decides to be root because has not received a synchronization message for ROOT\_TIMEOUT (lines 3 to 5).
% A node sends synchronization messages if it is root or has synchronized (lines 6 to 10).
% If a node is root, it also has to increment its \textit{highestSeqNum}. 


No pr�ximo cap�tulo, iremos apresentar o FTSP+, que � uma alternativa ao FTSP sem necessidade do recurso de \textit{timestamp} na camada MAC. Demostraremos os detalhes da t�cnica referente a estimativa do atraso no envio e o pseudoc�digo da implementa��o deste recurso.
 
