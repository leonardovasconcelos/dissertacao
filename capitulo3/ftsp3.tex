\section{Flooding Time Synchronization Protocol}\label{sec:ftsp}


O FTSP surgiu em 2004 \cite{maroti2004} com sua principal meta, efetuar a sincroniza��o dos rel�gios de todos os participantes de uma rede com boa performance, simplicidade e baixo \textit{overhead}. Assim, tendo que atender as restri��es que os componentes das RSSF possuem, onde todo n� tem um erro relativo a impureza do cristal, bem como, erros inerentes ao enlace de comunica��o sem fio n�o confi�vel.

Usando uma �nica mensagem o FTSP sincroniza m�ltiplos receptores, a mensagem com a marca de tempo � gravada no instante em que est� sendo enviada, que tamb�m � praticamente o mesmo momento em que esta sendo marcada como recebida pelo receptor. Este mecanismo de \textit{timestamp} na camada MAC elimina a maior parte das fontes de erros na etapa de comunica��o sumarizados na Se��o \ref{subsec:fontes_imp} e � empregado em muitos outros mecanismos \cite{Ganeriwal:Sensys03, Lenzen:Sensys09}. O trabalho \cite{elson2002} trata do uso de regress�o linear para a corre��o do escorregamento do rel�gio, que � o usado pelo FTSP para compensar o seu rel�gio e assim manter um alto n�vel de acur�cia.

O n� \textit{root} � o respons�vel por disseminar o tempo global pela rede, por�m h� redes em que os n�s n�o est�o ao alcance da faixa de cobertura do seu sinal. O FTSP cobre esse problema, pois tem o recurso de rede multi-saltos, ele constr�i uma estrutura \textit{ad hoc}, onde, os n�s diretamente sincronizados com o \textit{root} assim que ficam atualizados passam a sincronizar os n�s subsequentes com sua informa��o de tempo global. 
