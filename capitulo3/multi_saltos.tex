\section{Sincroniza��o Multi-saltos}

\outline{
\begin{itemize}
 \item Formato da mensagem de sincroniza��o.
 \item Gerenciamento de informa��o redundante.
 \item Elei��o do n� raiz.
 \item Evento: enviar/receber mensagem de sincronia. 
\end{itemize}

}




\begin{figure}[h]
 \begin{Verbatim}[numbersep=1pt,frame=single]
typedef nx_struct TimeSyncMsg
{
 nx_uint16_t rootID;
 nx_uint16_t nodeID; 
 nx_uint8_t  seqNum;
 nx_uint32_t globalTime;
 nx_uint32_t localTime;
} TimeSyncMsg;
\end{Verbatim}
\caption{Formato da mensagem de sincroniza��o}
\end{figure}




%This section briefly describes the Flooding Time Synchronization Protocol (FTSP). For more detailed information, refer to \cite{maroti2004}.

% FTSP is a synchronization protocol for WSN that provides high accuracy, consumes few resources, uses little bandwidth and is fault-tolerant. It elects a node as root to provide the time reference for synchronization; if root failure is detected (using timeouts), another root is elected.
% Root and synchronized nodes send synchronization messages periodically, and receiving nodes use these messages to synchronize.
% Therefore, FTSP supports multi-hop networks.
% 
% Synchronization messages comprise a \textit{sender timestamp} which is the estimated global time and \textit{rootID} which is the network identifier of the root (where the node with the lowest ID is the chosen root).
% \textit{seqNum} is a sequence counter that is incremented each synchronization round; this field is used to verify the redundancy of messages \cite{maroti2004}.
% 
% All nodes think they are root when the network starts, so they broadcast synchronization messages to the network. 
% When they receive a synchronization message, they check who has the lowest ID: if the local ID is higher, this node gives up on being root and starts synchronizing.
% Another important check is the \textit{seqNum}.
% If it is greater than the local value \textit{highestSeqNum}, it means that this is a new synchronization message and starts the synchronization procedure.
% 
% The synchronization procedure consists of computing a linear regression \cite{elson2003} that will provide the clock skew (used to estimate the global time) in relation to the reference node.
% The last step is to forward its local (synchronized) time to other nodes. 

\begin{figure}[h]
 \centering
 \includegraphics{./figuras/exemplo_ftps1.eps}
 % exemplo_ftps1.eps: 0x0 pixel, 300dpi, 0.00x0.00 cm, bb=0 0 330 180
 \caption{Exemplo do fluxo da rede com FTSP}
 \label{fig:exemplo_ftsp1}
\end{figure}


\begin{figure}[h]
\begin{Verbatim}[numbers=left,numbersep=1pt,frame=single]
event Radio.receive(TimeSyncMsg *msg){
  if( msg->rootID < myRootID )
      myRootID = msg->rootID;
  else if( msg->rootID > myRootID
    || msg->seqNum <= highestSeqNum )
      return;
  highestSeqNum = msg->seqNum;
  if( myRootID < myID )
      heartBeats = 0;
  if( numEntries >= NUMENTRIES_LIMIT
    && getError(msg) > TIME_ERROR_LIMIT )
      clearRegressionTable();
  else
      addEntryAndEstimateDrift(msg);
}
\end{Verbatim} 
 \caption{FTSP Receive Routine \cite{maroti2004} }%\vspace{0em}
 \label{figure1}
\end{figure}

% Figure 1 shows the routine of receiving synchronization messages.
% Lines 2 and 3 compare the \textit{rootID} of the synchronization message with the local \textit{rootID}.
% If the message has a lower \textit{rootID}, the node assumes this \textit{rootID} as root.
% Lines 4 to 7 ignore messages with higher  \textit{rootID} and lower \textit{seqNum}.
% If \textit{seqNum} is higher, local \textit{highestSeqNum} is updated.
% Lines 8 and 9 makes a root node give up on being root when it has a \textit{rootID} lower than its own ID.
% In case the \textit{rootID} is larger it is checked whether the \textit{seqNum} is greater or equal to the value of \textit{highestSeqNum}, this check prevents information redundancy because the message will only be used when the \textit{rootID} is less than or equal to \textit{myRootID} and the number greater than the value of \textit{highestSeqNum}.
% Lines 10 to 15 verify if the time of a message is in disagreement with the earlier estimates of global time, if applicable clear the regression table, if not, accumulate synchronization messages to calculate the linear regression and synchronize.


%Lines 10 to 15 accumulate synchronization messages to calculate the linear regression and synchronize.



\begin{figure}[h]
   \begin{Verbatim}[numbers=left,numbersep=1pt,frame=single]
event Timer.fired() {
  ++heartBeats;
  if( myRootID != myID
    && heartBeats >= ROOT_TIMEOUT )
      myRootID = myID;
  if( numEntries >= NUMENTRIES_LIMIT
    || myRootID == myID ){
      msg.rootID = myRootID;
      msg.seqNum = highestSeqNum;
      Radio.send(msg);
  if( myRootID == myID )
      ++highestSeqNum;
  }
}
  \end{Verbatim}
  \caption{FTSP Send Routine \cite{maroti2004}}
  \label{figure2}
\end{figure}

% Figure \ref{figure2} shows the sending routine.
% A node decides to be root because has not received a synchronization message for ROOT\_TIMEOUT (lines 3 to 5).
% A node sends synchronization messages if it is root or has synchronized (lines 6 to 10).
% If a node is root, it also has to increment its \textit{highestSeqNum}. 
