\chapter{Conclus�o e Trabalhos Futuros} 
\outlineon=0



Neste trabalho apresentamos uma vers�o modificada do \textit{Flooding Time Synchronization Protocol} que trabalha sem o \textit{timestamp} na camada de acesso ao meio. Para manter a acur�cia alta usamos interrup��es de r�dio para medir o instante que os n�s recebem o acesso ao meio. O emissor usa uma mensagem de corre��o, al�m da mensagem de sincroniza��o, como forma de compensa��o do atraso relativo ao acesso ao meio.

% In this work, we presented a modified version of Flooding Time Synchronization Protocol to work without MAC layer timestamping. To keep accuracy as high as possible, we use radio interrupts to measure the instant nodes get medium access. Senders use correction messages in addition to synchronization messages in order to compensate their synchronization timestamps with medium access delay.

Em nossos experimentos executamos testes em \textit{motes} Micaz rodando TinyOS 2.1.2 para medir a estimativa do tempo de corre��o, tempo de processamento e acesso ao meio. N�s mostramos que o tempo de acesso ao meio � a principal fonte erro de sincroniza��o, quantificamos os valores em um ambiente de testes real. Tamb�m mostramos que o tempo de processamento � muito baixo, na m�dia de $0.87$ \textit{jiffys}. Os resultados deste trabalho foram submetidos ao Simp�sio Brasileiro de Redes de Computadores e Sistemas Distribu�dos (SBRC'16).

% In our experiments, we ran tests on Micaz motes running TinyOS 2.1.2 to measure correction times, processing times and medium access time. We showed that medium access time is the main source of synchronization error and quantified it in a real testbed. We also showed that processing times are very small, on average $0.87$ jiffys. 

Existem trabalhos recentes, que procuram melhorar o desempenho do FTSP \cite{shannon2012dynamic}, outros est�o criando alternativas que implementam funcionalidades que s�o o ponto de desvantagem do FTSP \cite{Sommer:IPSN09, Lenzen:Sensys09}. No trabalho \cite{lenzen2015pulsesync}, os autores demonstram resultados melhores que o FTSP nos testes que realizaram, por�m todos esses mecanismos fazem uso de \textit{timestamp} na camada de acesso ao meio. Assim, a t�cnica desenvolvida no FTSP+, pode servir como trabalho futuro a inclus�o do recurso tamb�m a esses novos protocolos de sincroniza��o. 

%In future work, we want to precisely measure the synchronization error using a high accuracy external clock to measure timing error between synchronization events among the network motes.
